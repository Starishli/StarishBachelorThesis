%# -*- coding: utf-8-unix -*-
%%==================================================
%% chapter01.tex for SJTU Master Thesis
%%==================================================

%\bibliographystyle{sjtu2}%[此处用于每章都生产参考文献]
\chapter{绪论}
\label{chap:intro}

\section{研究背景}

期货和期权等衍生品是金融市场和商品市场上风险管理的重要工具。
对于资产管理机构,他们可以使用金融衍生品来管理自身的beta暴露,控制系统性风险;对于实体企业,他们也可以通过买卖商品衍生品,降低由于原材料或产成品价格波动过大而带来的经营风险。
期货作为一种线性资产,其价格变动与标的价格变动呈线性关系。
实际操作中,企业可以根据未来风险的方向,按照一个套期保值比例,在期货市场上买入或卖出期货合约。原理简单、操作方便是期货的优势。
然而,使用期货进行风险管理也存在一些问题。首先,期货为保证金交易,其杠杆比率较低,因此对资金占用较高,会给企业带来较高的机会成本。更重要的是,期货的线性特性使对冲者在规避了可能的损失的同时,也放弃了未来潜在的收益。

与期货相比,期权有着独有的优势,可以有效地帮助企业解决使用期货进行风险管理时的问题。若以买入期权的方式进行对冲,则对冲者只需在期初支付一定量的期权费,在对冲期间无需关注保证金账户。并且以期权费衡量的杠杆比率也较高,用较少的资金就可以实现和期货相同的对冲头寸数量。
同时,这一对冲方式风险有限(损失全部的期权费),在价格有利变动时也可以保留获取收益的能力。因此,相比之下,期权是一种更好的对冲工具。

根据交易方式的不同,衍生品又可以分为场内衍生品(Exchange-Traded Derivatives)和场外衍生品(Over-The-Counter Derivatives)。场内衍生品又称交易所衍生品,这一类衍生品由交易双方通过交易所以竞价的方式完成交易;场外衍生品又称柜台交易衍生品,这一类衍生品由交易双方直接或通过共同对手方(Central Conterparty)进行交易和结算。与场内市场相比,场外市场通常规模更大。张玉红(2006)指出,从上世纪90年代开始,美国的场外衍生品市场的增长速度就已明显高于场内市场。同时,由于场外衍生品市场可以为企业提供特殊定制的风险管理产品,而场内市场提供的是标准化的合约。因此,场外市场和场内市场在功能上并不完全同质化。斯文(2012)通过对美国的场外衍生品市场数据进行实证分析发现,场外衍生品市场和场内市场更多地呈现出替代关系。近年来,我国场外衍生品市场发展迅猛。根据中国证券业协会的数据,截至2018年7月31日,我国场外衍生品的初始名义本金累计3.27万亿元,存量2973.54亿元。

期权这一衍生品相比于期货,涉及到更多维度的变量,企业对其的定制化需求也就更高。
因此,相比场内期权市场,我国的场外期权市场更为活跃。
目前我国场内市场中,商品期权有白糖、棉花、豆粕、玉米、铜和天然橡胶六种,金融期权有上证50ETF期权一种;而场外市场中,期权基本覆盖了大部分交易所中交易的标的。我国场外期权的交易商为证券公司和期货公司。对于这些交易商来说,他们是期权流动性的提供者,一般处于净卖出期权的位置。
他们面临着两个很实际的问题:一是如何通过场内市场复制期权,对冲自身暴露的期权头寸;二是如何根据对冲的成本对卖出的期权进行定价,以在覆盖自身对冲成本的前提下提供一个更有竞争力的价格。

我国期权市场仍处于发展阶段,与成熟市场还有较大差距。场内市场期权品种较少,无法和品种丰富的场外市场匹配,这也给上文提到的两个问题的解决增加了难度。由于缺少场内非线性的工具,交易商只能够通过使用期货或现货等线性资产来进行Delta上的动态对冲。因此,基于目前我国期权市场的现状,本文将试图对以下问题做出解答:1)如何采取恰当的Delta动态对冲策略以及确定动态对冲中的相关参数,以获得最好的对冲效果;2)在实际操作中,交易成本将在多大程度上影响最终的对冲成本。

\section{研究意义}

本文以虚拟案例的形式,使用了模拟研究和实证分析两种方法,同时基于固定时点的动态对冲和固定Delta区间的动态对冲,对场外期权的定价和对冲进行了研究。

在模拟研究中,本文使用了几何布朗运动资产价格路径模型。并且根据期权价格的特点,以波动率溢价的形式来标准化交易成本对动态对冲结果的影响。在衡量不同参数下的对冲结果时,模仿风险厌恶函数提出了一种新的评判标准,用于实际应用中动态对冲参数和具体方法的选择。

在实证分析的研究中,本文使用动力煤指数的数据,进行了滚动的动态对冲分析,以为该期权未来的定价和对冲提供一定的参考,这一框架同样适用于基于其他标的的场外和场内期权的定价。同时,将实证分析的结果和模拟研究的结果比较,有助于定性地帮助场外期权交易商理解实际应用和理论分析的差异,以更好地将动态对冲方法应用到实际生产中。

\section{研究内容}

本文分析和论述结构如下:

第一章为绪论,简要介绍了本文的研究背景、研究意义和研究内容,提出了本文研究的整体框架。

第二章为文献综述,主要介绍了目前有关方向上的研究历史及现状,包括期权定价理论和动态对冲分析的国内外相关研究。

第三章为研究方法,主要介绍了期权定价和动态对冲的基本理论及实现方法,同时将对本文所用的模拟方法以及模拟中各个参数的选取和处理进行详细介绍。

第四章为模拟与实证分析,首先给出了一个虚拟的场外期权交易案例,之后使用蒙特卡洛模拟的方法以及固定时点动态对冲和固定Delta区间动态对冲两种动态对冲策略,进行对冲效果以及最优对冲参数选择的分析,最后使用动力煤指数的实际数据,与模拟研究类似地进行动态对冲的实证分析,并将其与模拟的结果进行比较。

\section{研究方法与研究思路}

本文试图在传统的期权动态对冲研究的基础之上,建立一个从模拟研究到实证分析的研究框架,并且通过比较两者的差异来获得实际应用中的启示。在模拟方法的确定中,我们主要通过考察不同维度下的收敛速度,来选择最适用于本文所述问题的模拟方法。在模拟研究中,本文使用固定的波动率和交易成本,考察固定时点的动态对冲和固定Delta区间的动态对冲两种对冲策略,并且使用对冲成本的期望的隐含波动率和对冲成本的标准化波动率作为评价标准。在实证分析中,本文基于模拟研究的思路,分别使用固定的隐含波动率和动态调整的波动率,考察动态对冲策略在实际数据上的表现。

\newpage
\section{研究框架}

\begin{figure}[!htp]
    \centering
    \resizebox{6cm}{!}{\begin{tikzpicture}[node distance=1.5cm]
    \node (intro) [process] {问题提出};
    \node (bib)  [process, below of=intro] {文献综述};
    \node (method) [process, below of=bib] {主要理论与方法};
    \node (sim) [process, below of=method] {模拟研究};
    \node (empirical) [process, below of=sim] {实证分析};
    \node (conclusion) [process, below of=empirical] {总结与展望};

    %连接具体形状
    \draw [arrow](intro) -- (bib);
    \draw [arrow](bib) -- (method);
    \draw [arrow](method) -- (sim);
    \draw [arrow](sim) -- (empirical);
    \draw [arrow](empirical) -- (conclusion);

\end{tikzpicture}
}
    \caption{研究框架}
    \label{fig:flow_chart}
\end{figure}
