%# -*- coding: utf-8-unix -*-
%%==================================================
%% chapter01.tex for SJTU Master Thesis
%%==================================================

%\bibliographystyle{sjtu2}%[此处用于每章都生产参考文献]
\chapter{文献综述}
\label{chap:bib}

\section{国外研究结果}

期权定价理论和动态对冲理论是金融学中经典的理论,两者的理论基础也较为相似。实际上,期权定价理论的推导本身就是基于了动态对冲的方法,因此期权定价和动态对冲可以说是一枚硬币的两面。然而,由于两者的侧重点仍有所不同:期权定价理论倾向于给出期权价格的解析形式或直接使用数值的方法计算出期权的价格,动态对冲理论倾向于给出一个最优的动态对冲策略及策略损益的分布,主要关注对冲中关键参数的确定。因此本章也将分别考察这两个方面的研究,并且将着重关注有交易费用时的相关研究。

\subsection{期权定价理论}

1973年,Black和Scholes提出了著名的BS模型(Black-Scholes期权定价模型),即在无套利原则和连续对冲假设下通过在Delta上的动态对冲推导出Black-Scholes方程,之后对方程进行变换后解出解析解,进而得出定价公式。BS模型提供了一个金融衍生品定价的基本框架,为各类金融衍生品的合理定价奠定了基础。之后,诸多学者对这一经典的定价模型进行修正,使其适用于各类金融衍生产品定价。1976年,Black根据CAPM模型和期货收益率分布的特点,对BS模型稍加改动,提出了用于期货期权定价的BS模型。Garman和Kohlhagen(1983)针对利率演化与BS模型中假设的不同,修改了“收益率符合对数正态分布”这一假设,提出了适用于外汇期权定价的模型。

然而,上述研究并没有并没有解决BS模型存在的一个关键问题,即其假设对冲是连续的且没有交易费用的。这一假设在实际市场中是几乎不可能成立的。Leland在1985年针对这一问题进行了研究,他通过修正BS模型中的波动率的方式,证明了在固定对冲时间间隔和有与标的价格成比例的对冲成本的情况下,对冲的误差与标的价格无关。实际上,这一方法的思路是在原有模型基础上增加一个波动性溢价,以抵消对冲成本的影响,他也给出了这一波动率溢价的解析形式。Leland的这一研究是非常有启发性的,之后很多的对于这一问题的研究都围绕其研究展开。如1992年,Boyle和Vorst基于Leland(1985)对波动率的修正,使用了二叉树模型进行数值模拟,得出了在一定参数取值范围内的较为精确的期权价格估计。

1993年,Davis etc.认为Leland(1985)给出的解析形式并不是在有交易费用时的期权定价的“最优”解(效用上的最优)。因此,他们基于Hodges和Neuberger(1989)的研究,使用效用最大化理论来考察这一问题。虽然在推导中使用了对数效用函数,但是他们认为定价结果与效用函数的具体形式无关。然而,Davis etc.提出的模型也有不足。主要存在的问题是在他们的模型中,期权价格是一个三维自由边界问题的解,其在计算上耗时较长,不符合实际应用中对价格计算速度的要求;其次,在价格中需要先验地给出投资者的效用函数,这一效用函数实际上较难确定。因此,Whalley和Wilmott(1997)在Davis etc.(1993)的研究的基础上,使用渐进分析的方法,将原有的三维自由边界问题转化为了一个二维扩散方程,提高了求解速度,使其可以在实际市场中得以应用,这一二维扩散方程实际上是在BS方程的基础上增加一个更小阶数的修正项。并且,他们由此推导出交易费用与BS模型下的Gamma有关。

\subsection{动态对冲理论}

动态对冲是风险管理的重要方法,同时也是期权定价推导的基础。一般而言,场外期权的动态对冲可以分为两种:一种是使用场内期权对冲场外期权暴露的头寸,一种是使用标的资产或资产组合对期权进行复制。在实务中,又以第二种方法使用较多,国外关于动态对冲的研究也大多基于使用线性资产对期权进行复制。

Black和Scholes(1973)在推导BS模型时,即是使用标的资产对期权在Delta上进行对冲。他们在每个时点计算期权的Delta,之后调整标的资产的头寸使其与期权暴露的Delta值匹配,再通过以无风险利率借入或贷出所需的资金,最后将每个时点产生的成本折现加总,即得到总对冲成本。当对冲时间间隔趋于0时,总对冲成本即为BS模型得出的期权价格。由此也可以看出动态对冲和期权定价之间的关系:动态对冲是期权定价理论推导的基础,反之期权定价得出的解析形式可以用于实务中动态对冲中Delta的确定。

从BS模型的框架中,可以看出动态对冲中的主要需要决定参数有两个:第一个是用于计算对冲头寸Delta,第二个是对冲的具体时间间隔,之后的扩展研究也基本围绕这两点展开。Black和Scholes的框架由于是相等时间间隔对冲,因此它其实是事先确定了第二个参数。在此基础之上对有交易费用时的动态对冲的修正,比如Leland(1985)的研究,也是同样地维持相等时间间隔不变,而对Delta的计算进行了修正,这一修正是通过修正计算BS公式下的Delta的波动率而实现的。关于这一对冲误差的收敛性,Leland认为当交易费用与标的价格成比例且保持不变时,随着对冲时间间隔趋于0,对冲误差也收敛于0,即对冲成本将从概率上收敛于修正的BS模型。然而,Kabanov和Safarian于1997年在Lott(1993)工作的基础上通过证明,得出了与Leland相反的结论:当交易费用不变时,随着对冲时间间隔趋近于0,对冲误差并不收敛于0;只有当交易费用和对冲时间间隔同时趋于0时,对冲误差才会收敛于0。2003年,Pergamenshchikov在Kabanov和Safarian的工作的基础上进一步证明了这一对冲误差的收敛速度为$n^{\frac{1}{4}}$,并给出了Leland组合终值的极限分布。之后,Darses和Denis(2010)更进一步地证明了对冲误差的极限分布。同时,他们通过修正Leland的模型以及采用不等间隔的对冲间隔,提高了对冲误差的收敛速度。

Leland及其之后的一系列研究大部分采用了固定的对冲时间间隔,主要关注用于计算Delta的波动率以及对冲误差的收敛性。与此不同的是,也有学者使用效用理论对动态对冲进行有交易费用时的动态对冲研究研究,通过效用最大化来寻找动态对冲中合适的参数,在这种情况下,对冲间隔并不能事先确定。这一类动态对冲策略需要在事先设定好决策规则之后,每个时点监控标的价格以及计算动态对冲参数,之后决定是否做出对冲调整。Hodges和Neuberger(1989)最早将基于效用的最优和期权复制联系在一起,他们使用了风险厌恶效用函数,通过构造一个偏微分方程来求解最优的对冲头寸水平。实际上,他们研究中的最优对冲策略主要考虑的是对Delta复制的准确性和交易费用之间的一个权衡,并且使用一个控制变量来定量地反映这一权衡的过程。当这一控制变量小于某一个临界值时,则不做出对冲头寸的调整,通过暴露一定的风险来节省交易费用。从对冲结果的标准差来看,这一对冲策略要优于Leland(1985)的对冲策略。在此基础上,Davis etc.(1993)证明了这一问题的解和效用函数的具体形式无关。之后,Whalley和Wilmott(1997)也证明了在存在交易费用时,Delta上的动态对冲存在无交易区间、买入区间和卖出区间。

除了以上基于对有交易费用时的动态对冲的修正的研究之外,另有一些动态对冲的研究关注了其他设定条件下的对冲组合及其相关评价指标。如Sepp(2013)基于BS模型以及四种不同的资产价格路径:对数正态扩散过程、跳跃扩散过程、随机波动率模型和带跳跃的随机波动率模型,首先给出了固定时点对冲方法下期望损益、期望对冲成本和损益波动率的解析式,并且由此给出了最优化夏普率时的对冲频率。Basak和Chabakauri(2012)的研究关注了不完全市场下最优动态对冲组合的解析形式。Shokrollahi和Sottinen(2017)基于Sottinen和Viitasaari(2017)在分形布朗运动上的研究,给出了分形市场假说下条件均值对冲组合的解析式。Hull和White(2017)则转而挖掘收益波动率和收益率之间的关系,探讨了在收益波动率和收益率相关的条件下的最小化组合波动率的对冲模型,并使用标普500的数据予以检验。

\section{国内研究结果}

由于我国衍生品市场起步较晚,期权类产品更甚,因此国内关于期权定价和动态对冲的研究较少。在已有的对于动态对冲的研究中,研究方法以蒙特卡洛模拟和实证研究为主,研究的内容则大多关注不同动态对冲策略效果的比较分析。张程(2010)通过构建工商银行的欧式看涨期权,使用实证的方法证明了固定Delta区间对冲策略优于固定时间对冲策略。熊辉(2015)和蒋论政(2018)在硕士论文中分别对玉米期权和豆粕期权的动态对冲策略做了分析和比较,也都得出了固定Delta区间对冲策略效果较好的结论。魏洁(2011)对股指期货和股指期权套期保值进行模拟,并对各套期保值方法进行了评价。卫剑波和王琦(2014)基于SLSG(Stop-Loss Start-Gain)对冲策略,使用沪深300的数据进行实证分析,发现对冲波动率和实现波动率的差异是对冲时收益的主要来源,并且以此为基础探索了Gamma识别和趋势识别的对冲策略。张卫国和杜谦(2016)使用蒙特卡洛方法对基于随机模型预测控制的对冲方法和传统的delta对冲方法进行了比较分析,并使用上证50ETF期权合约进行了实证检验,进而证明了基于随机模型预测控制的对冲方法的有效性。

\section{本章小结}

本章主要介绍了国内外有关期权定价和动态对冲的研究情况,重点介绍了有交易费用时的期权定价和动态对冲的研究。国外研究主要以模型建立为主,自1973年Black和Scholes提出BS模型以来,之后诸多学者从波动率修正和效用最大化等角度对有交易费用时的期权定价和动态对冲进行了研究。其中波动率修正角度的研究认为需要固定对冲的时间间隔,通过在原有波动率的基础上施加一个溢价来抵消交易费用带来的成本增加的影响;效用最大化角度的研究则关注在不完全对冲时的风险暴露和完全对冲时的额外交易成本之间的权衡,通过效用函数最大化来决定是否进行对冲,因此其对冲时间间隔并不固定。相比之下,国内这一方面的研究起步较晚,大多关注不同对冲策略的对比和评价,研究方法也以模拟分析和实证研究为主。
