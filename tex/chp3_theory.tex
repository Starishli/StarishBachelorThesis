%# -*- coding: utf-8-unix -*-
%%==================================================
%% chapter01.tex for SJTU Master Thesis
%%==================================================

%\bibliographystyle{sjtu2}%[此处用于每章都生产参考文献]
\chapter{主要模型和参数处理}
\label{chap:bib}

\section{期权定价理论}

\section{模拟方法}

在动态对冲研究中,模拟方法主要应用在资产价格路径的模拟上。关于资产价格路径的模拟,常用的有蒙特卡洛(Monte Carlo)方法和拟蒙特卡洛(Quasi-Monte Carlo)方法。蒙特卡洛方法是数值模拟中最为常用的方法,它实际上是这一类模拟方法的总称。本文所说的蒙特卡洛方法是一个相对狭义的概念,是指通过生成随机(random)或伪随机(pseudo-random)序列,模拟一个随机事件的可能情况,进而用频率来估计概率的方法。根据大数定律,若想有效地应用蒙特卡洛方法,需要较多的模拟的次数。本文并不直接考察每一次模拟的结果,而是对每一次的结果的一个函数进行求期望的操作,这一期望即为蒙特卡洛积分。设模拟次数为N,根据中心极限定理,蒙特卡洛积分的收敛速度为O(N-1/2)。蒙特卡洛的优势在于,其收敛速度独立于积分的维数。这一特点也使得其鲁棒性非常强,可以适用于很多高维的问题。然而,这一鲁棒性的代价是相对较慢的收敛速度。若要将误差的标准差大小的小数点向后移一位,需要将模拟次数提升为的100倍。

虽然蒙特卡洛方法鲁棒性较好,但是其对计算时间要求较高,因此本文最初希望可以找到一种收敛速度更快同时又不失鲁棒性的方法。我们考察了拟蒙特卡洛方法。拟蒙特卡洛方法使用低差异序列(low discrepancy sequence)进行模拟,经典的低差异序列包括Halton序列、Sobol序列和Faure序列。差异(discrepancy)是用来形容均匀性(uniformity)的,低差异序列比伪随机序列更接近均匀分布 。与蒙特卡洛方法使用线性同余法等方法通过生成伪随机序列来试图模仿随机数的性质不同,低差异序列实际上是确定性的序列,并且其具有一定的自相关性来降低差异。因此拟蒙特卡洛方法只能用在蒙特卡罗积分问题上,使用其进行优化或单纯考察其模拟的结果是无意义的。拟蒙特卡洛模拟的收敛速度是O((logN)dN-1),其中d为维数。从这一收敛速度可以看出,当模拟次数N相对于维数d很大时,可以获得接近O(1/N)的收敛速度。然而,当d增长时,N需要以指数速度增长,以维持相应的收敛速度。如果N不足够大的话,拟蒙特卡洛模拟的收敛速度会慢于蒙特卡洛模拟的收敛速度。正如Caflisch(1998)指出的,高维性会很大程度上限制拟蒙特卡洛模拟的有效性。具体到本课题的研究上,由于动态对冲是一个路径依赖的问题,因此其对应的维数即为标的资产价格路径模拟的频数,这一维数通常会很高(大于50)。在这样高维的模拟下,拟蒙特卡洛模拟效果将不如蒙特卡洛模拟。在之后的研究中,本文将使用蒙特卡洛模拟生成资产价格序列。

当然,蒙特卡洛模拟可以通过方差减少技术来加快收敛速度,这一速度上的提高通常是通过减小O(N-1/2)项的系数,而并不会将速度提升一个量级,对于这一技术,本文将不深入进行讨论,亦不会在应用中有所体现。关于拟蒙特卡洛模拟在高维情况下表现较差的问题,Wang and Sloan(2008)给出了一个解决方法,他们使用了一个新的计算差异的算法,使得拟蒙特卡洛方法在高维时的表现优于或至少不弱于蒙特卡洛方法。对于这一算法的细节、实现及应用,本文亦不进行讨论,而是将其作为未来可能的改进方向。

在确定了模拟使用的方法后,我们只是得到了生成参数为(0, 1)的均匀分布的随机数的方式。在获得了这一随机数后,可以使用正态分布的逆变换获得正态分布的随机数。之后,基于几何布朗运动和风险中性的假设,获得价格序列。
