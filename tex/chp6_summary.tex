%# -*- coding: utf-8-unix -*-
%%==================================================
%% chapter01.tex for SJTU Master Thesis
%%==================================================

%\bibliographystyle{sjtu2}%[此处用于每章都生产参考文献]
\chapter{总结与展望}
\label{chap:summary}

\section{结论总结}

在本文中,我们首先分析了动态对冲的理论基础以及动态对冲策略效果的影响因素。之后我们使用模拟研究和实证分析两种方法,分别基于固定时点动态对冲策略和固定Delta区间动态对冲策略,对无交易费用和有交易费用时的非连续动态对冲进行了分析和评价,并尝试探讨影响动态对冲效果的因素和比较分析模拟结果与实证结果的异同,以得出对场外期权交易商实际动态对冲操作的启示。

在模拟研究中,我们发现再平衡次数的提高在有效地降低对冲成本的波动的同时,也会由于交易费用的引入而提高期望对冲成本。基于我们推导的均值-方差对冲得分评判标准,降低对冲成本的波动对最终得分的正面影响要超过期望对冲成本增加带来的负面影响。在此基础上,我们得出了固定Delta区间对冲策略要优于固定时点对冲策略的结论,并且在给定的参数范围内,Delta阈值的选择取决于对冲者的风险厌恶程度。

在实证分析中,我们使用了动力煤指数的数据,进行了动态对冲策略的滚动回测。我们发现,实证分析和模拟研究结果上的差异主要在两方面:1)实证结果中无论是期望对冲成本还是对冲成本波动均高于模拟结果。这一差异既有实证研究中数据不足,带来的收敛性不佳的原因,更主要是因为实证分析中难以对未来波动率有准确的估计从而导致Delta计算的不准确,由此带来更高的对冲成本和对冲成本波动。2)实证结果中再平衡次数越低,期望对冲成本反而越高。这说明在实际操作中,对冲不精确对对冲成本的影响要大于一定范围内交易费用增加对对冲成本的影响。因此,虽然整体上看固定Delta区间对冲策略要优于固定时点对冲策略,这与模拟研究的结果相同并且可以同样地得到解释,但是在最优策略的选择上,我们选择了再平衡时间间隔为1天的固定时点对冲策略。

为了改进实证分析中的策略表现,我们使用动态波动率替代固定波动率。回测结果显示,使用动态波动率可以明显降低期望对冲成本,但是会略微提高对冲成本波动率。考察对冲得分后发现,动态波动率的结果优于固定波动率的结果,在两种动态波动率中又以动态EWMA波动率对未来波动率的预测效果最好。因此,场外期权交易商在对冲中可以使用动态波动率,这对对冲效果有明显的改善。

综上所述,从理论和模拟分析上,在综合考虑交易费用之后固定Delta区间动态对冲策略要优于固定时点动态对冲策略。但是,在实际市场操作中,对冲效果受交易费用影响较小,主要取决于对冲的精确程度。由于固定Delta区间动态对冲更依赖于人为估计的未来实际波动率的准确性,因此使用固定波动率时,再平衡频率较高的固定时点对冲策略更优。使用动态波动率可以提高对未来波动率估计的准确度,从而对两个对冲策略的对冲效果均有很大提升,并且对固定Delta区间对冲策略的提升更为明显。

\section{相关建议}

本文以动力煤指数的场外期权动态对冲为例,分析了动态对冲的影响因素,提出了从模拟研究到实证分析的完整研究框架。我们建立了对冲成本率和均值-方差对冲得分作为对冲结果的评价标准,并利用实际数据进行了动态对冲策略的评价以及对冲参数的选择。这一研究思路对于场外期权交易商的实际动态对冲操作有很强的指导意义。

本文的研究结果表明,在动态对冲中,虽然参数以及策略的选择千变万化,但是最关键的还是对标的资产未来波动率的预测能力。期权可以说是未来波动率的具象化代表,其收益特点与标的资产的未来波动率紧密相关。因此,对于场外期权交易商来说,如果可以准确地把握标的资产未来波动率的变化特点,那么就是把握住了期权的命门,在对冲中可以做到从心所欲不逾矩。

然而,在很多情况下,准确地预测波动率是一件很困难的事情。这时,就可以使用诸如动态波动率等方法,尝试刻画未来波动率的变化特点。实证结果表明,这一方法确实可以提升对冲效果,有助于对冲成本的降低。对于对冲中出现的对冲成本,场外期权交易商在定价时可以合理制定波动率溢价。更低的对冲成本意味着交易商可以提供给客户一个更有竞争力的价格,因此,交易商采用的动态对冲策略的表现将直接影响到其市场份额和盈利能力。我们建议场外期权交易商可以基于本文的研究框架,对各类标的资产的动态对冲策略进行系统性的研究,以对场外期权产品的对冲成本有一个直观的理解。通过选择更优的对冲策略和对冲参数,帮助自身降低成本,提高竞争力。
