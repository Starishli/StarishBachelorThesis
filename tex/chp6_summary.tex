%# -*- coding: utf-8-unix -*-
%%==================================================
%% chapter01.tex for SJTU Master Thesis
%%==================================================

%\bibliographystyle{sjtu2}%[此处用于每章都生产参考文献]
\chapter{总结与展望}
\label{chap:summary}

\section{结论总结}

在本文中,我们首先分析了动态对冲的理论基础以及动态对冲策略效果的影响因素。之后我们使用模拟研究和实证分析两种方法,分别基于固定时点动态对冲策略和固定Delta区间动态对冲策略,对无交易费用和有交易费用时的非连续动态对冲进行了分析和评价,并尝试探讨影响动态对冲效果的因素和比较分析模拟结果与实证结果的异同,以得出对场外期权交易商实际动态对冲操作的启示。

在模拟研究中,我们发现再平衡次数的提高在有效地降低对冲成本的波动的同时,也会由于交易费用的引入而提高期望对冲成本。基于我们推导的均值-方差对冲得分评判标准,降低对冲成本的波动对最终得分的正面影响要超过期望对冲成本增加带来的负面影响。在此基础上,我们得出了固定Delta区间对冲策略要优于固定时点对冲策略的结论,并且在给定的参数范围内,Delta阈值的选择取决于对冲者的风险厌恶程度。

在实证分析中,我们使用了动力煤指数的数据,进行了动态对冲策略的滚动回测。我们发现,实证分析和模拟研究结果上的差异主要在两方面:1)实证结果中无论是期望对冲成本还是对冲成本波动均高于模拟结果。这一差异既有实证研究中数据不足,带来的收敛性不佳的原因,更主要是因为实证分析中难以对未来波动率有准确的估计从而导致Delta计算的不准确,由此带来更高的对冲成本和对冲成本波动。2)实证结果中再平衡次数越高,期望对冲成本反而越高。这说明在实际操作中,对冲不精确对对冲成本的影响要大于一定范围内交易费用增加对对冲成本的影响。因此,虽然整体上看固定Delta区间对冲策略要优于固定时点对冲策略,这与模拟研究的结果相同并且可以同样地得到解释,但是在最优策略的选择上,我们选择了再平衡时间间隔为1天的固定时点对冲策略。

综上所述,从理论和模拟分析上,在综合考虑交易费用之后固定Delta区间动态对冲策略要优于固定时点动态对冲策略。但是,在实际市场操作中,对冲效果受交易费用影响较小,主要取决于对冲的精确程度。由于固定Delta区间动态对冲需要人为估计标的资产的未来实际波动率,这一估计的准确性将直接影响到对冲效果。因此,在不能够准确估计标的资产未来波动率时,使用再平衡频率较高的固定时点对冲策略更优。对于对冲中出现的对冲成本,场外期权交易商在定价时应当合理制定波动率溢价,这不仅可以降低最终的期望对冲成本,还可以降低对冲成本的波动。
