%# -*- coding: utf-8-unix -*-
%%==================================================
%% chapter01.tex for SJTU Master Thesis
%%==================================================

%\bibliographystyle{sjtu2}%[此处用于每章都生产参考文献]
\chapter{绪论}
\label{chap:intro}

\section{研究背景}

期货和期权等衍生品是金融市场和商品市场上风险管理的重要工具。
对于资产管理机构,他们可以使用金融衍生品来管理自身的beta暴露,控制系统性风险;对于实体企业,他们也可以通过买卖商品衍生品,降低由于原材料或产成品价格波动过大而带来的经营风险。
期货作为一种线性资产,其价格变动与标的价格变动呈线性关系。
实际操作中,企业可以根据未来风险的方向,按照一个套期保值比例,在期货市场上买入或卖出期货合约。原理简单、操作方便是期货的优势。
然而,使用期货进行风险管理也存在一些问题。首先,期货为保证金交易,其杠杆比率较低,因此对资金占用较高,会给企业带来较高的机会成本。更重要的是,期货的线性特性使对冲者在规避了可能的损失的同时,也放弃了未来潜在的收益。

与期货相比,期权有着独有的优势,可以有效地帮助企业解决使用期货进行风险管理时的问题。若以买入期权的方式进行对冲,则对冲者只需在期初支付一定量的期权费,在对冲期间无需关注保证金账户。并且以期权费衡量的杠杆比率也较高,用较少的资金就可以实现和期货相同的对冲头寸数量。
同时,这一对冲方式风险有限(损失全部的期权费),在价格有利变动时也可以保留获取收益的能力。因此,相比之下,期权是一种更好的对冲工具。

根据交易方式的不同,衍生品又可以分为场内衍生品(Exchange-Traded Derivatives)和场外衍生品(Over-The-Counter Derivatives)。场内衍生品又称交易所衍生品,这一类衍生品由交易双方通过交易所以竞价的方式完成交易;场外衍生品又称柜台交易衍生品,这一类衍生品由交易双方直接或通过共同对手方(Central Conterparty)进行交易和结算。与场内市场相比,场外市场通常规模更大。张玉红(2006)指出,从上世纪90年代开始,美国的场外衍生品市场的增长速度就已明显高于场内市场。同时,由于场外衍生品市场可以为企业提供特殊定制的风险管理产品,而场内市场提供的是标准化的合约。因此,场外市场和场内市场在功能上并不完全同质化。斯文(2012)通过对美国的场外衍生品市场数据进行实证分析发现,场外衍生品市场和场内市场更多地呈现出替代关系。近年来,我国场外衍生品市场发展迅猛。根据中国证券业协会的数据,截至2018年7月31日,我国场外衍生品的初始名义本金累计3.27万亿元,存量2973.54亿元。

期权这一衍生品相比于期货,涉及到更多维度的变量,企业对其的定制化需求也就更高。
因此,相比场内期权市场,我国的场外期权市场更为活跃。
目前我国场内市场中,商品期权有白糖、棉花、豆粕、玉米、铜和天然橡胶六种,金融期权有上证50ETF期权一种;而场外市场中,期权基本覆盖了大部分交易所中交易的标的。我国场外期权的交易商为证券公司和期货公司。对于这些交易商来说,他们是期权流动性的提供者,一般处于净卖出期权的位置。
他们面临着两个很实际的问题:一是如何通过场内市场复制期权,对冲自身暴露的期权头寸;二是如何根据对冲的成本对卖出的期权进行定价,以在覆盖自身对冲成本的前提下提供一个更有竞争力的价格。

我国期权市场仍处于发展阶段,与成熟市场还有较大差距。场内市场期权品种较少,无法和品种丰富的场外市场匹配,这也给上文提到的两个问题的解决增加了难度。由于缺少场内非线性的工具,交易商只能够通过使用期货或现货等线性资产来进行Delta上的动态对冲。基于上述现象,本文试图对以下问题做出解答:1)如何采取恰当的Delta动态对冲策略,以获得最好的对冲效果;2)在实际操作中,交易成本将在多大程度上影响最终的对冲成本。

\section{研究意义}

本文使用了蒙特卡洛模拟和实证分析的方法,基于固定时点的动态对冲和固定Delta区间的动态对冲,对场外期权的定价进行了研究。

在蒙特卡洛模拟的研究中,本文使用了几何布朗运动





\subsection{准备工作}
\label{sec:requirements}

要使用这个模板撰写学位论文,需要在\emph{TeX系统}、\emph{TeX技能}上有所准备。

\begin{itemize}[noitemsep,topsep=0pt,parsep=0pt,partopsep=0pt]
	\item {\TeX}系统:所使用的{\TeX}系统要支持 \XeTeX 引擎,且带有ctex 2.x宏包,以2015年的\emph{完整}TeXLive、MacTeX发行版为佳。
	\item TeX技能:尽管提供了对模板的必要说明,但这不是一份“ \LaTeX 入门文档”。在使用前请先通读其他入门文档。
	\item 针对Windows用户的额外需求:学位论文模本分别使用git和GNUMake进行版本控制和构建,建议从Cygwin\footnote{\url{http://cygwin.com}}安装这两个工具。
\end{itemize}

\subsection{模板选项}
\label{sec:thesisoption}

sjtuthesis提供了一些常用选项,在thesis.tex在导入sjtuthesis模板类时,可以组合使用。
这些选项包括:

\begin{itemize}[noitemsep,topsep=0pt,parsep=0pt,partopsep=0pt]
	\item 学位类型:bachelor(学位)、master(硕士)、doctor(博士),是必选项。
	\item 中文字体:fandol(Fandol 开源字体)、windows(Windows 系统下的中文字体)、mac(macOS 系统下的华文字体)、ubuntu(Ubuntu 系统下的文泉驿和文鼎字体)、adobe(Adobe 公司的中文字体)、founder(方正公司的中文字体),默认根据操作系统自动配置。
	\item 英文模版:使用english选项启用英文模版。
	\item 盲审选项:使用review选项后,论文作者、学号、导师姓名、致谢、发表论文和参与项目将被隐去。
\end{itemize}

\subsection{编译模板}
\label{sec:process}

模板默认使用GNUMake构建,GNUMake将调用latemk工具自动完成模板多轮编译:

\begin{lstlisting}[basicstyle=\small\ttfamily, caption={编译模板}, numbers=none]
make clean thesis.pdf
\end{lstlisting}

若需要生成包含“原创性声明扫描件”的学位论文文档,请将扫描件保存为statement.pdf,然后调用make生成submit.pdf。

\begin{lstlisting}[basicstyle=\small\ttfamily, caption={生成用于提交的学位论文}, numbers=none]
make clean submit.pdf
\end{lstlisting}

编译失败时,可以尝试手动逐次编译,定位故障。

\begin{lstlisting}[basicstyle=\small\ttfamily, caption={手动逐次编译}, numbers=none]
xelatex -no-pdf thesis
biber --debug thesis
xelatex thesis
xelatex thesis
\end{lstlisting}

\subsection{模板文件布局}
\label{sec:layout}

\begin{lstlisting}[basicstyle=\small\ttfamily,caption={模板文件布局},label=layout,float,numbers=none]
├── LICENSE
├── Makefile
├── README.md
├── bib
│   ├── chap1.bib
│   └── chap2.bib
├── bst
│   └── GBT7714-2005NLang.bst
├── figure
│   ├── chap2
│   │   ├── sjtulogo.eps
│   │   ├── sjtulogo.jpg
│   │   ├── sjtulogo.pdf
│   │   └── sjtulogo.png
│   └── sjtubanner.png
├── sjtuthesis.cfg
├── sjtuthesis.cls
├── statement.pdf
├── submit.pdf
├── tex
│   ├── abstract.tex
│   ├── ack.tex
│   ├── app_cjk.tex
│   ├── app_eq.tex
│   ├── app_log.tex
│   ├── chapter01.tex
│   ├── chapter02.tex
│   ├── chapter03.tex
│   ├── conclusion.tex
│   ├── id.tex
│   ├── patents.tex
│   ├── projects.tex
│   ├── pub.tex
│   └── symbol.tex
└── thesis.tex
\end{lstlisting}

本节介绍学位论文模板中木要文件和目录的功能。

\subsubsection{格式控制文件}
\label{sec:format}

格式控制文件控制着论文的表现形式,包括以下几个文件:
sjtuthesis.cfg, sjtuthesis.cls和GBT7714-2005NLang.bst。
其中,“cfg”和“cls”控制论文主体格式,“bst”控制参考文献条目的格式,

\subsubsection{主控文件thesis.tex}
\label{sec:thesistex}

主控文件thesis.tex的作用就是将你分散在多个文件中的内容“整合”成一篇完整的论文。
使用这个模板撰写学位论文时,你的学位论文内容和素材会被“拆散”到各个文件中:
譬如各章正文、各个附录、各章参考文献等等。
在thesis.tex中通过“include”命令将论文的各个部分包含进来,从而形成一篇结构完成的论文。
对模板定制时引入的宏包,建议放在导言区。

\subsubsection{各章源文件tex}
\label{sec:thesisbody}

这一部分是论文的主体,是以“章”为单位划分的,包括:

\begin{itemize}[noitemsep,topsep=0pt,parsep=0pt,partopsep=0pt]
	\item 中英文摘要(abstract.tex)。前言(frontmatter)的其他部分,中英文封面、原创性声明、授权信息在sjtuthesis.cls中定义,不单独分离为tex文件。
不单独弄成文件。
	\item 正文(mainmatter)——学位论文正文的各章内容,源文件是chapter\emph{xxx}.tex。
	\item 附录(app\emph{xx}.tex)、致谢(thuanks.tex)、攻读学位论文期间发表的学术论文目录(pub.tex)、个人简历(resume.tex)组成正文后的部分(backmatter)。
参考文献列表由bibtex插入,不作为一个单独的文件。
\end{itemize}

\subsubsection{图片文件夹figure}
\label{sec:fig}

figure文件夹放置了需要插入文档中的图片文件(支持PNG/JPG/PDF/EPS格式的图片),可以在按照章节划分子目录。
模板文件中使用\verb|\graphicspath|命令定义了图片存储的顶层目录,在插入图片时,顶层目录名“figure”可省略。

\subsubsection{参考文献数据库bib}
\label{sec:bib}

目前参考文件数据库目录只存放一个参考文件数据库thesis.bib。
关于参考文献引用,可参考第\ref{chap:example}章中的例子。
